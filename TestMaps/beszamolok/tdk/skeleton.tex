\documentclass{report}
\usepackage[utf8]{inputenc}
\usepackage{t1enc}
\usepackage{setspace}
\usepackage[magyar]{babel}
\usepackage{graphicx}
\usepackage{wrapfig}
\usepackage{enumitem}
\usepackage{listingsutf8}
\usepackage{hyperref}
\usepackage{ragged2e}
\usepackage{lmodern}

\begin{document}
	\tableofcontents
	\chapter{Bevezetés}
		\section{Motiváció}
			Napjainkra az emberek széles körének vált elérhetővé a gépkocsival való közlekedés. Sajnos amikor úthálózatunkat a múltszázad közepén megtervezték, még álmodni sem mertek a mérnökök a mai forgalom méretéről. Ez sajnos meg is látszik, az útjaink állandóan telítettek, mindennapossá váltak a kilométeres forgalmi torlódások.
			
			Mivel a torlódásokat, dugókat sokszor az emberi figyelmetlenség -- vagy egyszerűen csak a lassabb reakcióidő -- okozza, a járműgyártók igyekeznek minél több és minél kifinomultabb vezetéstámogató rendszert szerelni autóinkba. Ez a technológia kezd elég fejletté válni ahhoz, hogy a gépjárművek akár saját magukat is vezessék, akár emberi beavatkozás nélkül is. Másfél-két évtizeden belül valós elképzelésnek tűnik, hogy a mindennapi forgalom egy részét már ilyen, önvezető járművek adják.
			
			Ha ezen autókat felruházzuk megfelelő kommunikációs képességekkel, akkor minden lehetőség adott egy kooperatívan működő ágensrendszer kialakításához is. Lehetővé válhat általa az útfelület jobb kihasználása kisebb követési távolságokkal. Megnövelhető lenne a forgalmi csomópontok áteresztő képessége új, kommunikáción alapuló forgalomirányító mechanizmusokkal. A sávváltások forgalomcsökkentő hatása is csökkenthető lenne, ha a járművek egy csoportja váltana egyszerre sávot és nem egyesével kellene sávot váltania az egyes autóknak. Emellett persze nem mehetünk el olyan előnyök mellett sem, mint a forgalom biztonságának növekedése, vagy az üzemanyagfogyasztás -- ezzel együtt a károsanyag-kibocsátás -- csökkenése.
				
		\section{Elvégzendő kísérletek, elérendő célok}
			\begin{itemize}
				\item szűk egy oldal
				\item milyen kísérleteket, miért végzek el, mit várok tőlük
			\end{itemize}
		\section{Kapcsolódó irodalom}
			\begin{itemize}
				\item Kibővítem az eddigit
				\item Kb. másfél--két oldal
			\end{itemize}
		
	\chapter{Algoritmusok és kommunikációs protokolljavaslatok}
		\section{Elvi áttekintés}
			Az általam javasolt megoldások törekszenek használni a már meglévő közúti infrastruktúrát (pl. sávjelzések), ám szükséges lehet néhány új jelzés bevezetése. Ilyen jelzés lehet egy forgalmi csomópont megközelítését vagy éppen elhagyására figyelmeztető marker elhelyezése.
			
			Járműveink, hívjuk őket \emph{okosautóknak} ezen jelzések között \emph{csoportokba} tömörülnek Azok a járművek alkothatnak egy csoportot, akik a csomóponton azonos mozgási pályán kívánnak átmenni. Egy-egy csoportot a csoport élén haladó \emph{csoportvezető} vezeti, a többiek, a \emph{csoporttagok} pedig a csoportvezetőt követve próbálnak átkelni a kereszteződésen.
			
			A csoportvezetők döntései nyomán történnek a sávváltások is. Amennyiben egy csoportvezető úgy határoz, hogy a csoportja sávot fog váltani, ezt a szándékot közli a célsávban lévő csoport vezetőjével is. A két csoportvezető ezután megegyezik abban, hogy melyikük csoportja lesz a manőver végén előrébb. Ha szükséges, a célsávban lévő csoport beengedi maga elé a másik csoportot. 
		
			A csomópontot megközelítve pedig a csoportvezető jelenti be a csomópontban az áthaladási jogot osztó \emph{bíró} számára csoportja érkezését. A bíró saját logikája szerint az egyes csoportokat olyan ``nagyobb csoportokba'' sorolja, amely nagyobb csoportok egyszerre áthaladhatnak az adott csomóponton. Hívjuk ezeket a ``nagyobb csoportokat'' \emph{konfliktusosztálynak}! A bíró a konfliktusosztályok közül választja ki azt, aki éppen áthaladhat a csomóponton.
			
			Vegyük észre, hogy a bírók hasonló dolgot csinálnak, mint az operációs rendszerek ütemezői: egy szűkös erőforrás elosztásán dolgoznak egymással ``versengő'' kliensek számára. Innen jött az ötlet, hogy a már jól ismert algoritmusokat (némi módosítással) a bírók konfliktusosztály-váltó logikájaként fel lehetne használni.
			
		\section{Definíciók}
			\subsection{Statikus infrastrukturális elemek}
				\subsubsection{Belépési marker}
					A \emph{belépési marker} egy néhányszor tízméter hosszú szakaszt\footnote{A pontos hosszt nem érdemes definiálni. Azt az aktuális topológia befolyásolja ugyanis.} jelöl a csomópontba befelé vezető oldalon, de a csomóponttól viszonylag messzebb kezdve.
					
					A belépési marker jelzi az okosautók számára, hogy össze kell állni csoportokba.
					
				\subsubsection{Bejelentési távolság}
					A bejelentési távolság a csoportvezető számára hordoz információt. Ekkora távolságból kell ugyanis bejelentkeznie a bírónál. A bíró ekkor határozza meg, hogy egy csoport melyik konfliktusosztályba tartozik.
					
				\subsubsection{Megállási távolság}
					Ez gyakorlatilag a Kreszből ismert STOP-vonal megjelenítése. Technikai okokból a csomóponttól nagyjából 10~méteres távolságra van.\footnote{A preempció miatt előfordulhat olyan eset, hogy egy járműnek relatíve intenzíven kell fékeznie ahhoz, hogy ne fusson bele a csomópontba...}
				
				\subsubsection{Kritikus pont}
					A csomóponttól egy-két méter távolságra elhelyezett pont. Amelyik autó ezen a ponton áthalad, annak a csomóponton is minél előbb át kell haladnia. Az okosautók mindig bejelentik, ha áthaladnak ezen a ponton, így tudja a bíró, hogy éppen van-e autó a csomópontban.
					
				\subsubsection{Kilépési marker}
					Amint egy autó áthaladt a csomóponton, a kilépési markerhez ér. Itt egyfelől jelzi a bíró számára, hogy áthaladt a csomóponton. Másrészről a jármű kilép a csoportjából és konfliktusosztályából is.
					
					Mivel az csoport élén halad a csoportvezető, ezért természetesen ő lép ki először a csoportból. Ilyenkor gondoskodunk az új csoportvezető kiválasztásáról is.
					
				\subsubsection{Magyarázat a megállási távolság és kritikus pont létére}
					A mindennapi közlekedésünk során is elő-előfordul az a jelenség, hogy hiába észleljük időben a sárga lámpát, már nem tudunk biztonsággal megállni a lámpa előtt, ezért továbbhaladunk. Eközben elképzelhető, hogy a lámpa már pirosra is váltott.
					
					Ilyen esemény természetesen egy intelligens közlekedési rendszerben is bekövetkezhet. Csakhogy itt az algoritmusok érzékenyebbek arra, hogy éppen van-e valaki a kereszteződésben, akinek már elvileg nem lenne szabad ott lennie. Valószínűleg bonyolult implementációval ez a helyzet is számítható lenne, de ennél egyszerűbb megoldás, ha a következők szerint járunk el:
					\begin{enumerate}
						\item Csak bizonyos idő (pl. 3~s) letelte után indítjuk el a következő konfliktusosztályt. Ez nagyjából a sárga lámpa esete.
						\item Csak akkor indítjuk el az új konfliktusosztályt, ha már letelt az előző pontban említett idő, és a kereszteződés is üres. Ez pedig a pirosba belecsúszó autó esete.
					\end{enumerate}
					
					A megállási távolság és a kritikus pont egymáshoz való helyzetének megértéséhez szükséges a szimulációs platform, a SUMO (bővebben lásd \aref{sec:sumo}.~fejezetet) felépítésének ismerete. Tömören fogalmazva itt az autók egy gráf élein (ezek nagyjából a sávok) mozognak. A gráf pontjai pedig a sávok kapcsolódási pontjai, viszont a SUMO-ban ezek fizikai kiterjedéssel nem rendelkeznek. Tehát egy kereszteződés képe a SUMO-ban egy grafikai trükk, a csomópont pontos határainak ismerete így nehézkes volna. Ezért nem esik egybe a megállási távolság és a kritikus pont, ugyanis nem volna egyszerű rájuk egy konkrét értéket meghatározni. Ez a mi szempontunkból jelenleg nem is érdekes kérdés.\footnote{A kritikus pont, a megállási és bejelentési távolságok egy-egy koncentrikus kör mentén helyezkednek el, melyek középpontja a SUMO csomópontjának pozíciója.}
					
			\subsection{Okosautók és képességeik}
				\begin{itemize}
					\item fél és egy oldal közötti terjedelemben
				\end{itemize}
			\subsection{Csoportosítások definíciói}
				\subsubsection{Csoportok}
					\begin{itemize}
						\item kb. fél oldalban
					\end{itemize}
				\subsubsection{Konfliktusosztályok}
					\begin{itemize}
						\item ez is kb. fél oldalban
					\end{itemize}
			\subsection{Bírók}
				\begin{itemize}
					\item kb. fél oldalban
					\item a statikus elemekkel vett kapcsolatok leírása
				\end{itemize}
		\section{Csoportosítás}
			\subsection{Csoportformálás}
				\begin{itemize}
					\item 4-5 bekezdés
					\item protokoll elvi leírása (pontos leírást inkább függelékként írnám)
				\end{itemize}
			\subsection{Csoport képességei és előnyei}
				\begin{itemize}
					\item nagyjából féloldalas leírás
					\item főleg a csoportosítás által adott előnyökről szeretnék írni:
						\subitem kevesebb kommunikáció csomópontoknál
						\subitem együtt haladás
						\subitem sávváltások megkönnyítése
				\end{itemize}
			\subsection{Csoportok felbontása}
				\begin{itemize}
					\item a csoportformálásho hasonlóan 4-5 bekezdés, absztrakt protokolleírással
				\end{itemize}
		\section{Sávváltások}
			\subsection{A SUMO sávváltásának ismertetése}
				\begin{itemize}
					\item Van egy publikációja Krajzewicz-nek arról a sávváltási modellről, amit átírtam ``intelligensre'', itt erről szeretnék írni kb. egy oldalnyi összefoglalást
					\item Ebbe egy kisebb (már meglévő) protokoll ismertetését is szeretném beleszőni
				\end{itemize}
			\subsection{A módosítások ismertetése}
				\begin{itemize}
					\item Egy újabb egy-másfél oldalnyi terjedelemben
					\item írnék arról, hogy a már meglévő algoritmust, protokoll hogyan és miért módosítottam
				\end{itemize}
	\section{Bírók}
		\subsection{Csomóponti áthaladás protokollja}
			\begin{itemize}
					\item Kb. egyoldalas leírás a következőkről:
			\end{itemize}
			\subsubsection{Csatlakozás konfliktusosztályhoz}
			\subsubsection{Áthaladás a csomópontokon}
		\subsection{Forgalomirányítási algoritmus}
			\subsubsection{Döntés áthaladásról}
				\begin{itemize}
					\item kb. fél oldal
					\item aktív konfliktusosztály
					\item sárga (biztonsági) fázis szükségessége
				\end{itemize}
			\subsubsection{Konfliktusosztály váltása}
				\begin{itemize}
					\item kb fél oldal terjedelemben
					\item mikor kell konfliktusosztályt váltani
				\end{itemize}
		\subsection{Konfliktusosztály-váltási algoritmusok}
			\subsubsection{Rokonság operációs rendszerek ütemezéjővel}
				Nagyjából 2-3 bekezdés
			\subsubsection{Preemptív Round Robin ütemező}
				Néhány bekezdésnyi terjedelemben írnám le itt. A pontos algoritmust függelékként közölném.
			\subsubsection{MDDF ütemező}
				Nagyjából egy-másfél oldalnyi terjedelemben ismertetném az MDDF algoritmust Fahwadék publikációja alapján. Pár bekezdésben pedig kitérnék egy-két kisebb módosításáról, amit csináltam rajta.
				
	\chapter{Szimulációs platform}
		\label{sec:sumo}
		\section{A SUMO rövid bemutatása}
			\begin{itemize}
				\item kb. egy-két oldalas összefoglaló
				\item mi ez, hogyan épül fel
			\end{itemize}
		\section{A SUMO-n végzett bővítések}
			\begin{itemize}
				\item ide a nyári irományból szeretném kiragadni a lényeget
				\item kb. 3-4 oldalban
				\item kiegészítve azokkal a ``tervezői döntésekkel'', amik a SUMO működéséből adódnak: mágikus konstansokkal, furcsaságok leírásával
			\end{itemize}
	
	\chapter{Mérések}
		\section{Mérési elrendezések}
			\subsection{Mérések pályái}
				\begin{itemize}
					\item leírom a tesztpályát
					\item a BAH-csomópontot
					\item kb. féloldalas szöveges leírás + ábrák
				\end{itemize}
			\subsection{Lámpaprogramok}
				\begin{itemize}
					\item egy-két bekezdésben leírni a Bp. Közúttól kapott programokat
					\item valószínűleg ezek másodpercre optimalizáltak
				\end{itemize}
			\subsection{Normális és rendhagyó forgalom}
				\begin{itemize}
					\item Mit értek ezek alatt
					\item A BAH-ban mit nevezhetnénk rendhagyónak
					\item Itt jó kérdés, hogy mi alapján jelentjük ki, hogy mi a normális és mi a rendhagyó. Mert eddig empirikus alapokon, becsült forgalomértékekkel dolgoztam\dots
				\end{itemize}
			\subsection{(Intelligens) lámpavezérlés}
				\begin{itemize}
					\item unalmas leírás a konkrét BAH-csomópontbeli beállításokról
					\item MDDF, RR, hagyományos, szabályozatlan esetek leírása
					\item Ez is nagyjából egy oldalnyi lesz
				\end{itemize}
		\section{Mérendő mennyiségek}
			\begin{itemize}
				\item nagyjából féloldalas leírás
				\item megtett utak (megérkezett autók)
				\item átlagsebességek
				\item balesetek
				\item fogyasztások vagy károsanyagkibocsátások
				\item és ezek értelmezése
			\end{itemize}
		\section{Mérési eredmények}
			2-3 oldalnyi táblázatok, grafikonok
		\section{Mérési eredmények értékelése}
			\begin{itemize}
				\item Nagyjából másfél-két oldalon át elemzem az adatokat
				\item várt és kapott értékek közötti eltérések, hasonlóságok elemzése
			\end{itemize}
	\chapter{Összefoglalás, konklúzió}
		\begin{itemize}
			\item kb. 1 oldalon
		\end{itemize}
	\chapter{Függelék}
		\section{Pontos protokolleírások}
			\begin{itemize}
				\item Üzenetek, szekvenciadiagramok az egyes üzenetváltásokra
				\item Esetleg rövid magyarázat nem triviális megoldásokra
				\item protokollonként fél-egy oldal közötti terjedelem
			\end{itemize}
			\subsection{Csoportosítás}
				\subsubsection{Csoportba belépés}
				\subsubsection{Csoportból kilépés}
				\subsubsection{Csoportvezető váltása}
			\subsection{Sávváltás}
			\subsection{Bíróval kapcsolatos üzenetváltás}
				\subsubsection{Konfliktusosztályba belépés}
				\subsubsection{Bíró pollingozása}
		\section{Algoritmusleírások}
			\subsection{P-szabályozók}
				\begin{itemize}
					\item Van két P-szabályozón alapuló megoldásom a programban
					\item ezekről a SUMO-s fejezetben szeretnék bővebben írni
					\item itt részletezném, hogy mi a konkrét szabályozó algoritmus, mi a munkapontjuk stb.
				\end{itemize}
				\subsubsection{Adaptív tempomat csoporttagok között}
				\subsubsection{Megállás bírónál}
			\subsection{Preemptív RR ütemező}
				\begin{itemize}
					\item A konrkét konfliktusosztály-választó algoritmus pszeudokódbeli leírása
				\end{itemize}
			\subsection{MDDF ütemező}
				\begin{itemize}
					\item Az MDDF és az eredeti, és az én megvalósításom közötti eltérések leírása
					\item pszeudo-kód
				\end{itemize}
\end{document}